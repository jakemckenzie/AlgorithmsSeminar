\documentclass[12pt]{article}
\usepackage{amsmath,amssymb,amsthm}
\usepackage{graphicx,mathabx}
\usepackage{xcolor}
\usepackage{tikz}
\usepackage{placeins}
\usepackage{lipsum}
\usepackage[shortlabels]{enumitem}
\usepackage{placeins}
\usepackage[makeroom]{cancel}
\newcommand\tab[1][1cm]{\hspace*{#1}}
\begin{document}
\title{TCSS 343 - Week 5}
\author{Jake McKenzie}
\maketitle
\noindent\centerline{\textbf{Dynamic Programming}}\\\\\\\\\\\\\\\\
\begin{center}
    ``Perhaps thinking should be measured not by what you do \\but by how you do it." \\$\dots$\\ Richard Hamming
\end{center}
\begin{center}
    ``For the last sixty five years(speaking in 2018), due to Moore's law, with a clockwork precision, computer capability has been doubling every year and a half. Without fast algorithms you cannot bring to bare Moore's Law. A dramatic increase in computer speed needs to be coupled with efficient algorithms." \\$\dots$\\ Christos Papadimitriou
\end{center}
\begin{center}
    ``The best programs are the ones written when the programmer is supposed to be working on something else." \\$\dots$\\ Melinda Varian
\end{center}
\newpage
\noindent 1. I found these really cool recursive Fibonacci formulas: 
$F_{2n+1}=F_{n+1}^{2}+F_{n}^{2}$ and $F_{2n}=2F_{n+1}F_{n}-F_{n}^2$
 now can you use them to find $F_{11}$ and $F_{10}$(worth noting that
 $F_2 = 1$, $F_1 = 1$, and $F_0 = 0$)?\\\\\\\\\\\\\\\\\\\\
 2. Can you now write an iterative method that computes these 
 using dynamic programming?\\\\\\\\\\\\\\\\\\\\
 3. How is this different from the basic Fibonacci formula? What is the
 time complexity of this function? Space complexity?
\end{document}