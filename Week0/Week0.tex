\documentclass[12pt]{article}
\usepackage{amsmath,amssymb,amsthm}
\usepackage{graphicx,mathabx}
\usepackage{xcolor}
\usepackage{tikz}
\usepackage{placeins}
\usepackage{lipsum}
\usepackage[shortlabels]{enumitem}
\begin{document}
\title{TCSS 343 - Week 0}
\author{Jake McKenzie}
\maketitle
\noindent\centerline{\textbf{Recursion}}\\\\
1. Benin is a fisherman who is simply good at fishing. One day, he finds a nice place to go fishing with two ponds. 
Moving from the $i-th$ fish-pond (the one he starts at) to the $j-th$ fishpond would cost $|i - j|$ units of time. 
Initially Benin can get $F_i$ fish in the $i-th$ fishpond. 
In the next turn at the same fishpond, the amount of fish he can get is decreased by $D_i$. 
Notice that Benin will not get negative amount of fish.
Each turn of fishing takes Benin 1 unit of time if Benin is at that pond and $|i - j|$ units of time to switch.
\\\\
For example, if $F_1 = 10$, $F_2 = 5$, $D_1 = 2$, $D_2 = 3$ and Benin can fish for up to eight units of time, then he will get $10 + 8 + 6 + 5 + 4 = 33$.
Washington Department of Fish and Wildlife (WDFW) requires that Benin switch to the adjacent pond when it has more fish and he cannot fish for "negative" fish.
Write a recursive algorithm to see how many fish Benin can fish for!
\newpage
\end{document}