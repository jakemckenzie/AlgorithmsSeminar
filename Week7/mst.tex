\documentclass[11pt]{article}

\usepackage{fullpage}
\usepackage{epsfig}


\begin{document}

\title{{\bf Minimum Spanning Trees} \\
\normalsize{(CLRS 23)}}
\date{}

\maketitle

%%%%%%%%%%%%%%%%%%%%%%%%%%%%%%%%%%%%%%%%%%%%%%%%%%%%%%%%%%%%%%%%%%%%%%%%%%

\begin{itemize}
\item Problem: Given connected, undirected graph $G=(V,E)$ where each
  edge $(u,v)$ has weight $w(u,v)$. Find acyclic set $T \subseteq E$
  connecting all vertices in $V$ with minimal weight \\ $w(T) =
  \sum_{(u,v) \in T} w(u,v)$.
\item An acyclic set connecting all vertices is called a {\em spanning
  tree}. We want to find a spanning tree of {\em minimal weight}. We
  use {\em minimum spanning tree} as short for {\em minimum weight
  spanning tree}).
\item MST problem has many applications
  \begin{itemize}
  \item For example, think about connecting cities with minimal
    amount of wire or roads (cities are vertices, weight of edges
    are distances between city pairs).
  \end{itemize}
\item Example: \\
  \centerline{\epsfig{file=mst_example.eps,height=3.5cm}}
  \begin{itemize}
  \item Weight of MST is $4+8+7+9+2+4+1+2=37$
  \item MST is not unique: e.g. $(b,c)$ can be exchanged with $(a,h)$ 
  \end{itemize}
\end{itemize}


\section{PRIM's algorithm}
\begin{itemize}
\item {\em Greedy} algorithm for computing MST:
        \begin{itemize}
        \item Start with spanning tree containing arbitrary vertex $r$ and
        no edges
        \item Grow spanning tree by repeatedly adding minimal weight edge
        connecting vertex in current spanning tree with a vertex not in the
        tree
        \end{itemize}
\item Implementation:
  \begin{itemize}
  \item To find minimal edge connected to current tree we maintain a
    priority queue on vertices not in the tree. The key/priority of a
    vertex is the weight of minimal weight edge connecting it to the
    tree. (We maintain pointer from adjacency list entry of $v$ to $v$
    in the priority queue).
  \item For each node $u$ maintain $visit(u)$ ($(u, visit(u))$ is the
    cuurently best edge connecting it to the tree.)
  \end{itemize}
  
  \fbox{
    \parbox{10cm}{
      PRIM(r)\\
      
      For each $v \in V$ DO
      \begin{itemize}
      \item[] {\sc Insert}($PQ,v,\infty$)
      \end{itemize}
	  {\sc Decrease-Key}$(PQ,r,0)$ \\
	  WHILE $PQ$ not empty DO
          \begin{itemize}
          \item[] $u$ = {\sc Deletemin}($PQ$)
	  \item [] (output edge $(u, visit(u))$ as part of MST)
          \item[] For each $(u,v) \in E$ DO
            \begin{itemize}
            \item[] IF $v \in PQ$ and $w(u,v) < $ key($v$) THEN
            \item[] ~~~~visit[$v$] = $u$
            \item[] ~~~~{\sc Decrease-Key}($PQ,v,w(u,v)$)
            \end{itemize}
          \end{itemize}
  }}

\item On the example graph, the greedy algorithm would work as follows
	(starting at vertex $a$): \\
  
  \centerline{\epsfig{file=mst_prim_1.eps,height=7cm}}
  \centerline{\epsfig{file=mst_prim_2.eps,height=7cm}}
  

\item Analysis: 
  \begin{itemize}
  \item While loop runs $\vert V \vert$ times $\Rightarrow$ we
    perform $\vert V \vert$ {\sc Deletemin}'s
  \item We perform at most one {\sc Decrease-Key} for each of the $\vert E
    \vert$ edges \\
    $\Downarrow$ \\
    $O((\vert V \vert + \vert E \vert) \log \vert V \vert)=O(\vert E
    \vert \log \vert V \vert)$ running time.
  \end{itemize}
\item Correctness: 
  \begin{itemize}
  \item When designing a greedy algorithm the hard part is to
    prove that it works correctly.
  \item We will prove a Theorem that allows us to prove the
    correctness of a general class of greedy MST algorithms:
    
    Some definitions
    \begin{itemize}
    \item A {\em cut} $(S, V \setminus S)$ is a partition
      of $V$ into sets $S$ and $V\setminus S$
    \item A {\em edge $(u,v)$ crosses a cut $S$} if $u
      \in S$ and $v \in V\setminus S$ {\em or} $v \in S$ and $u
      \in V\setminus S$
    \item A {\em cut $S$ respects a set $T \subseteq E$}
      if no edge in $T$ crosses the cut
    \end{itemize}
    
    Example: Cut $S$ respects $T$ \\
    \centerline{\epsfig{file=cut_example.eps,height=5cm}}
  \end{itemize}
\item {\em Theorem}: If $G=(V,E)$ is a graph such that $T \subseteq
  E$ is subset of some MST of $G$, and $S$ is a cut respecting $T$ {\bf
    then} there is a MST for $G$ containing $T$ and the minimum weight edge
  $e=(u,v)$ crossing $S$.
  
\item Note: Correctness of Prim's algorithm follows from the Theorem by
  induction---cut consist of current spanning tree.

\item Proof:
  \begin{itemize}
  \item Let $T^*$ be MST containing $T$
  \item If $e \in T^*$ we are done
  \item If $e \notin T^*$:
    \begin{itemize}
    \item There must be (at least) one other edge $(x,y) \in T^*$
      crossing the cut $S$ such that there is a unique path from $u$
      to $v$ in $T^*$ ($T^*$ is spanning tree) \\
      \centerline{\epsfig{file=cut_proof.eps,height=6cm}}
    \item This path together with $e$ forms a cycle
    \item If we remove edge $(x,y)$ from $T^*$ and add $e$
      instead, we still have spanning tree
    \item New spanning tree must have same weight as $T^*$
      since $w(u,v) \leq w(x,y)$ \\
      $\Downarrow$ \\
      There is a MST containing $T$ and $e$.
    \end{itemize}
  \end{itemize}

\item The Theorem allows us to describe a very abstract greedy algorithm
  for MST: \\ \\
  \fbox{
    \parbox{8cm}{
      $T = \emptyset$ \\
      While $\vert T \vert \leq \vert V \vert - 1$ DO
      \begin{itemize}
      \item[] Find cut $S$ respecting $T$
      \item[] Find minimal edge $e$ crossing $S$
      \item[] $T = T \cup \{ e \}$
      \end{itemize}
  }} 
  \begin{itemize}
  \item Prim's algorithm follows this abstract algorithm.
  \item Kruskal's algorithm is another implementation of the abstract
  algorithm.
  \end{itemize}
\end{itemize}





\section{Kruskal's Algorithm}
\begin{itemize}
\item Kruskal's algorithm is another implementation of the abstract algorithm.
\item Idea in Kruskal's algorithm:
  \begin{itemize}
  \item Start with $|V|$ trees (one for each vertex)
  \item Consider edges $E$ in increasing order; add edge if it
    connects two trees
  \end{itemize}
  


\item Example: \\ \\
  \centerline{\epsfig{file=mst_kruskal_1.eps,height=7cm}} \\ \\
  \centerline{\epsfig{file=mst_kruskal_2.eps,height=7cm}} \\

\item Implementation: \\ \\
  We need (Union-Find) data structure that supports:
  \begin{itemize}
  \item {\sc Make-set}($v$): Create set consisting of $v$
  \item {\sc Union-set}($u,v$): Unite set containing $u$ and set
    containing $v$
  \item {\sc Find-set}($u$): Return unique representative for
    set containing $u$
  \end{itemize}
  
  \fbox{
    \parbox{10cm}{
      KRUSKAL \\
      
      $T = \emptyset$ \\
      FOR each vertex $v \in V$ {\sc Make-Set}($v$)
      
      Sort edges of $E$ in increasing order by weight \\
      FOR each edge $e=(u,v) \in E$ in order DO
      \begin{itemize}
      \item[] IF {\sc Find-Set}($u$) $\neq$ {\sc Find-Set}($v$) THEN
	\begin{itemize}
	\item[] $T = T \cup \{ e \}$
	\item[] {\sc Union-Set}($u,v$)
	\end{itemize}
      \end{itemize}
  }}

\item Analysis: 
  \begin{itemize}
  \item We use $O(|E|\log |E|)$ time to sort edges and we perform
    $\vert V \vert$ {\sc Make-Set}, $\vert V \vert -1$ {\sc Union-set},
    and $2 \vert E \vert$ {\sc Find-Set} operations.
  \item We will discuss a simple solution to the \emph{Union-Find
    problem} such that {\sc Make-Set} and {\sc Find-Set} take $O(1)$
    time and {\sc Union-Set} takes $O(\log V)$ time amortized.
    
    $\Downarrow$
    
    Kruskal's algorithm runs in time $O(|E|\log |E| + |V|\log
    |V|)=O((|E|+|V|)\log |E|) = O(|E|\log |V|)$ like Prim's algorithm.
  \end{itemize}

\item Correctness 
  \begin{itemize}
  \item follows from Theorem above: If minimal edge connects two trees
  then there exists a cut respecting the current set of edges (cut
  consisting of vertices in one of the trees)
  \end{itemize}
  

\end{itemize}




\section{Union-Find}
\begin{itemize}
\item The \emph{Union-Find problem}: Maintain a set system under:
  \begin{itemize}
  \item {\sc Make-set}($v$): Create set consisting of $v$
  \item {\sc Union-set}($u,v$): Unite set containing $u$ and set
    containing $v$
  \item {\sc Find-set}($u$): Return unique representative for
    set containing $u$
  \end{itemize}
  
\item Simple solution:
  \begin{itemize}
  \item Maintain elements in same set as a linked list with each
    element having a pointer to the first element in the list (unique
    representative) \\
    

Example: \\

\centerline{\epsfig{file=union-find_linked.eps,height=5cm}}

\item {\sc Make-Set}($v$): Make a list with one element
  $\Rightarrow O(1)$ time
\item {\sc Find-Set}($u$): Follow pointer and return unique
  representative $\Rightarrow O(1)$ time
\item {\sc Union-Set}($u,v$): Link first element in list with
  unique representative {\sc Find-Set}($u$) after last element
  in list with unique representative {\sc Find-set}($v$) $\Rightarrow
  O(|V|)$ time (as we have to update all unique representative
  pointers in list containing $u$)
  \end{itemize}
\item With this simple solution the $\vert V \vert -1$ {\sc Union-Set}
  operations in Kruskal's algorithm may take $O(\vert V \vert ^2)$ time.
\item We can improve the performance of {\sc Union-Set} with a very simple
  modification: Always link the smaller list after the longer list
  ($\Rightarrow$ update the pointers of the smaller list)
  \begin{itemize}
  \item One {\sc Union-Set} operation can still take $O(\vert V
    \vert)$ time, but the $\vert V \vert -1$ {\sc Union-Set} operations
    takes $O(\vert V \vert \log \vert V \vert)$ time altogether (one
    {\sc Union-Set} takes $O(\log \vert V \vert)$ time
    \emph{amortized}):
    \begin{itemize}
    \item Total time is proportional to number of unique
      representative pointer changes
    \item Consider element $u$:
      
      After pointer for $u$ is updated, $u$ belongs to a list of
      size at least double the size of the list it was in before \\
      $\Downarrow$ \\
      After $k$ pointer changes, $u$ is in list of size at least
      $2^k$ \\
      $\Downarrow$ \\
      Pointer can be changed at most $\log \vert V \vert$ times.
    \end{itemize}
  \end{itemize}
\item With improvement, Kruskal's algorithm runs in time $O(|E|\log |E| +
  |V|\log |V|)=O((|E|+|V|)\log |E|) = O(|E|\log |V|)$ like Prim's
  algorithm.
\end{itemize}


 

\end{document} 
