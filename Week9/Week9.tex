\documentclass[12pt]{article}
\usepackage{amsmath,amssymb,amsthm}
\usepackage{graphicx,mathabx}
\usepackage{xcolor}
\usepackage{tikz}
\usepackage{placeins}
\usepackage{lipsum}
\usepackage[shortlabels]{enumitem}
\usepackage{placeins}
\usepackage[makeroom]{cancel}
\usepackage{mathrsfs}
\usepackage{nicefrac}
\newcommand\tab[1][1cm]{\hspace*{#1}}
\def\blankpage{%
      \clearpage%
      \thispagestyle{empty}%
      \addtocounter{page}{-1}%pdf
      \null%
      \clearpage}
\begin{document}
\title{TCSS 343 - Week 10}
\author{Jake McKenzie}
\maketitle
\noindent\centerline{\textbf{Final Review}}\\\\\\\\\\\\
\begin{center}
    ``Behind every argument is someone's ignorance." \\$\dots$\\ Louis Brandeis
\end{center}
\begin{center}
    ``The fastest algorithm can frequently be replaced by one that is almost as fast and much easier to understand" \\$\dots$\\ Douglas W. Jones
\end{center}
\newpage
\newpage
Given an even number of coins in a row of arbitrary denominations, two players take turns taking a coin from either end. 
Winner is who gets the most cash. Player 1 goes first. 
0. Would you rather go first or second? Does it matter?\\\\\\\\\\\\\\\\
1. Assume that you go first, describe an algorithm to compute the maximum amount of money you can win 
with tidy input and output conditions.\\
(HINT: If you go first, is there a strategy you can follow which prevents you from losing? 
Try to consider how it matters when the number of coins are odd vs. even.)\\\\
\newpage
The Institute of Technology of UW Tacoma will offer eight courses and seminar cycles 
during the Summer quarter. Their starting and ending times cannot be changed, and are 
given in the following table:\\
% Please add the following required packages to your document preamble:
% \usepackage{graphicx}
\begin{table}[]
\resizebox{\textwidth}{!}{%
\begin{tabular}{|c|c|c|c|c|c|c|c|c|}
\hline
Course ID: & TCSS 440 & \multicolumn{1}{l|}{TCSS 445} & TCSS 450 & TCSS 456 & TCSS 487 & TCSS 491 & \multicolumn{1}{l|}{TCSS 372} & \multicolumn{1}{l|}{TCSS 343} \\ \hline
Starts At: & 8:15 am & 9 am & 9:45 am & 11 am & 12 pm & 1:15 pm & 1:45 pm & 2 pm \\ \hline
Ends At: & 10 am & 11 am & 10:45 pm & 12:15 pm & 2 pm & 2 pm & 3:45 pm & 4 pm \\ \hline
\end{tabular}%
}
\end{table}
Using the Interval Scheduling algorithm, determine:\\
2. The maximum number of courses/seminars that can share one single classroom.\\\\\\\\\\\\
3. The minimum number of classrooms needed to accommodate all these courses/seminars. \\\\\\\\\\\\
4. A suitable assignment of classrooms to courses/seminars matching the constraints found in the previous items.
\newpage
Al Gore-\textit{ithm} has become concerned about global warming and has decided to modify his energy consumption in order 
to minimize the amount of carbon dioxide ($CO_2$) produced.  There are three electricity providers in his area, but 
all three providers, depending on seasonal weather and various other factors, change their electricity generating 
methods from week-to-week.  Thus, for some weeks of the year company A might produce less $CO_2$ and for other weeks 
companies B or C might produce less $CO_2$.  Al would like to switch providers each week to make use of the one that 
produces the least $CO_2$, but switching creates extra $CO_2$ (e.g., the companies have to send trucks out to his house).\\\\


Here's an example using just two companies. Suppose that changing providers costs $40$ units of $CO_2$ and that, over two weeks, 
using company A will cause the release $130$ and $77$ units of $CO_2$ while using company B will cause release of $85$ and $101$ units 
of $CO_2$.  If we use company B for the first week and then change to company A for the second week then the total released is 
$85 + 77 + 40 = 202$.  If we stay with company B for the whole two weeks then the total released is $85 + 101 = 186$.\\\\

5. So here's the problem: We have lists, $a_1$, $a_2$, ..., $a_n$ and $b_1$, $b_2$, ..., $b_n$ and $c_1$, $c_2$, ..., $c_n$, of $CO_2$ amounts over $n$ weeks for 
companies A, B, and C, respectively.  We also know $p$, the amount of $CO_2$ released due to changing providers.  Give an algorithm 
to determine which provider to use for each week so that total $CO_2$ released is minimized.\\\\
6. Suppose that company A has switched to trucks that burn only bio-fuels. Now the amount of $CO_2$ released when switching to 
company A is reduced (but the switch-over cost remains the same for the other companies).  Briefly explain how your algorithm 
should be changed to reflect this new  information.
\newpage
For each of the following statements, respond \textit{True}, \textit{False}, or \textit{Unknown}\\\\
7. P $=$ NP\\\\\\\\
8. If a problem is NP-Complete then it is NP-Hard.\\\\\\\\
9. No problem in NP can be solved in polynomial time.\\\\\\\\
A. If there is a polynomial time algorithm to solve problem A then A is in P.\\\\\\\\
B. If there is a polynomial time algorithm to solve problem A then A is in NP.\\\\\\\\
\end{document}